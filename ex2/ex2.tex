\documentclass{article}
\usepackage[utf8]{inputenc}

\usepackage{titling}
\usepackage{cjhebrew}

\usepackage{enumerate}
\renewcommand{\labelenumii}{\theenumii}
\renewcommand{\theenumii}{\theenumi.\arabic{enumii}.}

\usepackage{amssymb}
\usepackage{graphicx}
\usepackage{adjustbox}

\newcommand{\then}{\longrightarrow}
\renewcommand{\iff}{\longleftrightarrow}

%%% Title = נושאים מתקדמים בבינה מלאכותית - תרגיל 2
\title{2 \<trgyl> - \<nw/s'ym mtqdmym bbynh ml'kwtyt>}
%%% Author = דני רובינסקי - 304359276
\author{304359276 - \<dny rwbynsqy>}
\date{}
\begin{document}

\maketitle

\begin{enumerate}
	\item { }
		\begin{enumerate}
			\item $D \then (B \land C)$
			\item $C \then (\lnot A \land \lnot B)$
			\item $D \iff (C \land \lnot A)$
			\item $(D \land \lnot C) \then A$
			\item $(\lnot D \then C) \land (D \then \lnot B)$
			\item $A \then ((\lnot B \land \lnot C) \then D)$
			\item $((A \land B \land C) \iff \lnot D) \land ((\lnot A \land \lnot B) \then (D \then C))$\\\\
		\end{enumerate}


	\item We'll define the following language:\\
			Constants: $\{Bill, Analysis, Geometry\}$\\
			Predicates: $\{Student, Smart, Loves, Takes\}$\\
			Functions: $\emptyset$\\
			Variables: $\{x,y\}$
	\begin{enumerate}
		\item $\forall x (Student(x) \then Smart(x))$
		\item $\exists x (Student(x))$
		\item $\exists x (Student(x) \land Smart(x))$
		\item $\forall x (Student(x) \then  \exists y (Student(y) \land Loves(x,y)))$
		\item $\forall x (Student(x) \then \exists y (\lnot(x=y) \land Student(y) \land Loves(x,y)))$
		\item $\exists x (Student(x) \then \forall y (Student(y) \then Loves(y,x)))$
		\item $Student(Bill)$
		\item $(Takes(Bill, Analysis) \lor Takes(Bill, Geometry)) \land \lnot (Takes(Bill, Analysis) \land Takes(Bill, Geometry))$
		\item $Takes(Bill, Analysis) \land Takes(Bill, Geometry)$
		\item $\lnot Takes(Bill, Analysis)$
		\item $\forall x (Student(x) \then \lnot Loves(x, Bill))$\\
		\end{enumerate}
                \newpage


	\item Visualization of the given kripke model: \begin{minipage}[t]{0.45\linewidth}
                          \raggedright
                          \adjustbox{valign=t}{%
                            \includegraphics[scale=0.75]{advai3}%
                          }
                          \medskip
                      \end{minipage}
	\begin{enumerate}
		\item Holds, because $1 \in V(p)$.
		\item Holds, because $2R2$ and $2 \in V(p)$.
		\item Holds, because there is no world x, such that 3Rx and hence the statement $\forall x \in W, 3Rx \then x \in V(p)$ is true.
		\item Doesn't hold, because $3 \notin V(p) \Rightarrow M,2\nVdash\square p$
		\item Holds, since 2 is the only image of 1 in R, we get that this statement holds iff $M,2\Vdash\Diamond p$, which holds, because $2R2 \land 2 \in V(p)$
		\item Doesn't hold, because the only neighbor of 1 is 2 and $2 \in V(p)$, so $M,2\nVdash p$.
		\item Holds, because $2R3 \land 3 \notin V(p)$ and hence $M,3\Vdash \lnot p$.
		\item Holds. First note that $(p\then p)$ is tautology, so the only verification we need is that $M,2\Vdash\Diamond(p\then p)$. But this means that there is some neighbor of 2 for which the same tautology holds and it holds for any world. Since 2 has neighbors (itself, or 3), both parts of the formula hold in 2 and hence the entire formula holds.
		\item Doesn't hold. Using the same logic as above, we'd reach to a conclusion that in order for the formula to be true in 3, there must be at least one neighbor of 3. Since that's not the case, the formula is false.\\\\
	\end{enumerate}
        \newpage


	\item I'm assuming this reads $(\Diamond p) \land (\square \lnot p)$: If the first part is true - there is at least one neighbor not in V(p) and the second part can't be true.\\
	Otherwise, if the formula reads as $\Diamond (p \land (\square \lnot p))$, then the following model $(W=\{1,2\}, R=\{(1,2)\}, V(p)={2})$ satisfies the formula.\\\\


	\item { }
	\begin{enumerate}
		\item $(W=\{1,2\}, R=\{(1,2)\}, V(A)=\{1\})$
		\item Valid. Proof: Given a world w, in which the first part, $\lnot\Diamond A \land \Diamond B$ is true means that $\forall v \in W, wRv \then v \notin V(A) \land \exists u \in W, wRu \land u \in V(B)$. Hence for u, $\lnot A \land B$ is true and so $\Diamond (\lnot A \land B)$ is true in w. This is correct for any $w \in W$ that satisfies the first part. For any w in which the first part if false - the entire formula is always true (due to the properties of the $\then$ operator).
		\item $(W=\{1,2,3\}, R=\{(1,2),(2,2),(2,3)\}, V(A)=\{3\}$.\\
			$M,1\Vdash\square\Diamond A$, but $M,1\nVdash\Diamond\square A$, because 2R2 and $2 \notin V(A)$.\\\\
	\end{enumerate}


        \item If R is equivalent, then $\forall w,w' \in W wRw' \iff w'Rw$. Assume $w \in V(\phi)$ for some $w \in W$, then for every $w' \in W$ s.t. $wRw'$, $\Diamond\phi$ holds, because $w'Rw$ and $w \in V(\phi)$. Hence, $\square\Diamond\phi$ holds for every such w, which proves the claim.
\end{enumerate}
\end{document}

